
%%% Local Variables:
%%% mode: latex
%%% TeX-master: t
%%% End:

\documentclass{article}
\usepackage{CJKutf8}
\usepackage{color}

\begin{document}
\begin{CJK}{UTF8}{gbsn}


\title{Vmmr 系统研发文档}
\author{Yong Gao\\
       Yuhang District, Hangzhou\\
       \emph{gaoyongwyl@163.com}}
\date{2015/04/14}
\maketitle


\begin{abstract}
 对实验系统进行记录。包括实验框架、实验流程和注意事项等内容。
\end{abstract}


\section{System work flow}
\label{sec:introduction}
The main steps to training and test vmmr dnn model are listed as the following:
 
\begin{enumerate}
  \item \textbf{解压数据。多个压缩包的话直接向同一目录解压,免去之后的数据拷贝}。
  \item \textbf{生成训练和验证集的list,及验证集的GroundTruth(\color{red}{TODO item})}。调用python脚本CreateLabelTrainTest.py。该脚本的使用说明如下:\\
     \begin{quotation}
      Create make and makemodel label dict, and all , train and test label list\\
     \textbf{ Positional arguments}:
      \begin{itemize}
        \item SampleFolder: DataSet's data foler, Sample top folder.\\
                          e.g.: ~/Projects/Vmmr/Data/V1\_AAuMColor/V1\_AAuMColor \\
      \end{itemize}         
      \textbf{optional arguments}: \\
      \begin{itemize}
        \item  -h, --help:       show this help message and exit \\
        \item  -r TrainTestRatio, -TrainTestRatio: (Default=24) TrainTestRatio. Ratio between the num of train and test \\
        \item  -sh IsRandShuffle, -RandomShuffle IsRandShuffle:(Default=1,yes) Whether shuffle list randomly. Must be true 
                                for multi-round exp or test. 0 denote not\\
        \item  -tgn TrainTestGroupNum, -TrainTestGroupNum TrainTestGroupNum: (Default=1)Train and test label list group number \\ 
      \end{itemize}
      \textbf{Example}: \\
      \textit{\color{blue}{CreateLabelTrainTest.py $\sim$/Projects/Vmmr/Data/V1\_AAuMColor/V1\_AAuMColor}}
     \end{quotation}
       
  \item \textbf{生成测试集的list}。调用CreateTestList.py,生成测试集的list。所有测试集在,数据集目录的Test文件夹下。生成的list文件放在Test文件夹下的\_List目录中。
               GroundTruth文件也会自动生成。
        CreateTestList.py脚本的使用说明如下:\\
    \begin{quotation}
       Create relpath file name list \\
       \textbf{positional arguments}:\\
       \begin{itemize}
       \item SampleFolder:  Sample top folder, in next level is make model folder. E.g.: \textit{ $\sim$/Projects/VehicleRecog/Data/V1\_AAuMColor/Test }
       \end{itemize}
       \textbf{optional arguments}:
       \begin{itemize}
       \item -h, --help:    show this help message and exit
       \end{itemize}    
       \textbf{Example}:\\
       \textit{\color{blue}{CreateTestList.py $\sim$/Projects/VehicleRecog/Data/V1\_AAuMColor/Test}}
       
     \end{quotation}
     
   \item \textbf{生成训练数据LevelDB}。调用convert\_imagesetex.bin / exe,产生一个或多个区域的杨本数据LevelDB。
       convert\_imagesetex应用程序使用说明如下:\\
       \begin{quotation}
         Convert a set of images to the leveldb format used as input for Caffe. Usage:\\
         \textit{\color{blue}{convert\_imagesetex DataSetNameVer(eg. V0, PreprocTypeIDs(sep by":") PatchIDs(sep by ":") NewWidth AugmentNum(<AugTrain:AugTest>) FuncCode(TestOrTrain, 0:testonly,1:tainonly,2: both) IsNewLdb} }\\
        \textbf{positional arguments}:\\
        \begin{itemize}
          \item DataSetNameVer:(eg. V0) data set name;
          \item PreprocTypeIDs:(sep by":") 0 is Color, 1 is gray, 2 is gray hist equal, 3 is color hist equal;
          \item PatchIDs:(sep by ":") -1 is vface, 4 is LeftHLamp, ... 10 is MidLineBot.
          \item NewWidth:Normalized vehicle face  witdh before cropping;
          \item AugmentNum: (<AugTrain:AugTest> increase how many samples by transformation, such as adding noise, change color. 0 means not to augment.
          \item FuncCode:  TestOrTrain(0:testonly,1:tainonly,2: both) 
          \item IsNewLdb: 1 denote creating new db. 0 denote append data to an existing db.
        \end{itemize}
        \textbf{Example}:\\
        \textit{\color{blue}{convert\_imagesetex.bin V1 0:1 -1:6 380 0:0 2 1}}
       \end{quotation}

    \item \textbf{计算样本的均值}。这里使用训练样本的leveldb作为caffe工具compute\_image\_mean的输入。为方便调用,DNNMakemodel\_PrepareData.py对该工具进行了封装,
                 其用法如下:\\
         \textit{usage: DNNMakemodel\_PrepareData.py [-h] [-ttlid TrainTestListID] DATASET\_NAME PATCH\_ID NewWidth}
         \textbf{Example}:\\
         \textit{\color{blue}{./DNNMakemodel\_PrepareData.py V1\_AAuMColor 6 150}}

     \item \textbf{\color{red}{数据准备检查}}\\
               \begin{itemize}
               \item 在生成list,类别label等文件同时,CreateLabelTrainTest.py还会调用\\GenVMMTruthFromFileName.py生成所有文件的GroundTruth信息文件。
                     注意检查该文件 是否已生成,否则测试时会因找不到该文件而报错,中止运行。\\
               \item 一定要检查生成的make-label和makemodel-label的字典中,make及makemodel是否按顺序排列,否则可能会影响性能。
               \item 检查convert\_imageset 是否转换了预期数目的文件到LevelDB数据库中。可以通过LdbImgViewer.bin或者查看转换的Log文件。
               \end{itemize}
      
    \item \textbf{训练或者fine tune 网络}。老版本的train\_net和finetune\_net现在被统一的caffe命令集替代。\\
       \textit{ caffe train --solver=... [--snapshot=...]}\\
       \textit{ caffe train --solver=... [--weights=...]}\\
       新的caffe命令集说明如下:\\
       caffe.bin: command line brew \\
       usage: \textit{ caffe <command> <args>} \\
       \\
       \textbf{Commands}:\\
       \begin{itemize}
        \item train:  train or finetune a model
        \item test:   score a model
        \item device\_query: show GPU diagnostic information
        \item time:  benchmark model execution time        
       \end{itemize}
       \textbf{ Flags from tools/caffe.cpp}:
       \begin{itemize}
        \item -gpu (Run in GPU mode on given device ID.) type: int32 default: -1
        \item -iterations (The number of iterations to run.) type: int32 default: 50
        \item -model (The model definition protocol buffer text file..) type: string default: ""
        \item -snapshot (Optional; the snapshot solver state to resume training.) type: string default: ""
        \item -solver (The solver definition protocol buffer text file.) type: string default: ""
        \item  -weights (Optional; the pretrained weights to initialize finetuning. Cannot be set simultaneously with snapshot.) type: string default: ""
        \end{itemize}
        
     \item \textbf{在标准测试集上进行测试}

     \item \textbf{根据每个车型的样本数目生成数据子集} 将原始的数据集按每个车型的样本数条件,生成子集。比如:生成每个车型样本数大于等于8的子集;生成每个车型样本数小于8的子数据集。\\
       通过调用$ShowClassNumHist.py$和$ExtractSubDataSet.py$两个脚本来完成。主要分三个步骤:\\
       \begin{itemize}
       \item 调用$ShowClassNumHist.py$生成类别-数量列表文件,该文件中安样本数量排序。
       \item 手工拷贝或删除,生成符合样本数量条件的类别-数量列表文件。
       \item 调用$ExtractSubDataSet.py$,以上一部生成的类别-数量文件作为参数之一,建立符合样本数量条件的新数据集。
       \end{itemize}
       

\end{enumerate}

\section{Mistakes Usually Made}
\begin{itemize}
\item 类别数目出错。比如:网络模型中的类别和实际类别不一致
\item 数据转换不对或不完全。只有一部分数据转换进了LevelDB,即只有一部分数据参与训练,效果肯定不好!
\item val-proto网络模型文件中的batch\_size决定实际的内存分配和运算,影响速度。在应用层一次送进一张识别的时候,应该设为$1$。否则,有无用的计算产生!
\item 

\end{itemize}


\end{CJK}
\end{document}
%%% Local Variables:
%%% mode: latex
%%% TeX-master: t
%%% End:
